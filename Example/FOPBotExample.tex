\documentclass[
    12pt,
    a4paper,
    ngerman,
    leqno
    ]{article}

% Packages
\usepackage[T1]{fontenc}
\usepackage[ngerman]{babel}
\usepackage{../FOPBot}
\setlength{\parindent}{0pt}

\begin{document}

% Begin of Example
\section*{FOPBot Examples}
Simple Example: \bigskip
\begin{FOPBotWorld}{3}{3}
    % Walls (Since Node anchors are for some reason not as exact we use coordinates)
    \draw[very thick] (.5,-.5) -- (.5,1.5);
    \draw[very thick] (1.5,.5) -- (1.5,2.5);
    % Coins
    \putcoin{1}{2}{2}
    \putcoin{1}{0}{2}
    \putcoin{1}{0}{2}
    % Bots
    \path (1,1) pic {Trianglebot};
    \path (0,2) pic[rotate=270] {Trianglebot=red};
    \path (2,2) pic[rotate=180] {Trianglebot=green};
    \path (2,0) pic[rotate=90] {Trianglebot=yellow};
    \path (0,0) pic[] {Trianglebot=violet};
\end{FOPBotWorld}
\bigskip\par
Example using Loops and a figure:
\begin{figure}[h]
    \centering
    \begin{FOPBotWorld}{11}{5}
        \foreach \x/\y in {
                {0/0},
                {0/1},
                {0/2},
                {0/3},
                {0/4},
                {1/4},
                {2/4},
                {1/2},
                {2/2},
                {4/0},
                {4/1},
                {4/2},
                {4/3},
                {4/4},
                {5/4},
                {6/4},
                {6/3},
                {6/2},
                {6/1},
                {6/0},
                {5/0},
                {8/0},
                {8/1},
                {8/2},
                {8/3},
                {8/4},
                {9/4},
                {10/4},
                {10/3},
                {10/2},
                {9/2},
            }{
                \putcoin{\x}{\y}{1}
            }
        \path (5,2) pic {Trianglebot};
    \end{FOPBotWorld}
    \caption{Gewünschtes Ergebnis}
\end{figure}
\vspace{1cm}\\
Example using the defined lables: \bigskip
\begin{FOPBotWorld}{3}{3}
    % Walls (Since Node anchors are for some reason not as exact we use coordinates)
    \draw[very thick] (.5,-.5) -- (.5,1.5);
    \draw[very thick] (1.5,.5) -- (1.5,2.5);
    % Coins
    \putcoin{1}{2}{2}
    \putcoin{1}{0}{2}
    \putcoin{1}{0}{2}
    % Bots
    \path (1,1) pic {Trianglebot};
    \path (0,2) pic[rotate=270] {Trianglebot=red};
    \path (2,2) pic[rotate=180] {Trianglebot=green};
    \path (2,0) pic[rotate=90] {Trianglebot=yellow};
    \path (0,0) pic[] {Trianglebot=violet};
    % Using the defined Lables
    \draw[ultra thick, {Latex}-, orange] (n-1-1.center) -- ++ (2cm,0) node[right, font=\sffamily]{center};
\end{FOPBotWorld}
\clearpage
Tetris Example:\bigskip
\begin{FOPBotWorld}{6}{8}
    %\node[] at (0,0) {\includegraphics[height=.85cm]{orange.png}};
    \path (0,0) pic[rotate=90] {Tetrisbot=yellow};
    \path (0,1) pic[rotate=90] {Tetrisbot=yellow};
    \path (0,2) pic[rotate=90] {Tetrisbot=yellow};
    \path (1,2) pic[rotate=180] {Tetrisbot=green};
    \path (2,2) pic[rotate=180] {Tetrisbot=green};
    \path (2,3) pic[rotate=180] {Tetrisbot=green};
    \path (3,3) pic[rotate=180] {Tetrisbot=green};
    \path (3,4) pic {Tetrisbot=red};
    \path (3,5) pic {Tetrisbot=red};
    \path (2,5) pic {Tetrisbot=red};
    \path (4,4) pic {Tetrisbot=red};
    \path (1,0) pic[rotate=180] {Tetrisbot=blue};
    \path (2,0) pic {Tetrisbot=aqua};
    \path (3,0) pic {Tetrisbot=aqua};
    \path (2,1) pic {Tetrisbot=aqua};
    \path (3,1) pic {Tetrisbot=aqua};
    \path (4,0) pic[rotate=270] {Tetrisbot};
    \path (4,1) pic[rotate=270] {Tetrisbot};
    \path (4,2) pic[rotate=90] {Tetrisbot=purple};
    \path (5,1) pic[rotate=90] {Tetrisbot=purple};
    \path (5,2) pic[rotate=90] {Tetrisbot=purple};
    \path (5,3) pic[rotate=90] {Tetrisbot=purple};
\end{FOPBotWorld}\bigskip\par
Tetris Example 2 (With coordinates):\bigskip
\begin{FOPBotWorld}{3}{2}

    \path (0,1) pic {Tetrisbot=red};
    \path (1,1) pic {Tetrisbot=red};
    \path (1,0) pic {Tetrisbot=red};
    \path (2,0) pic {Tetrisbot=red};
    % Koordinaten
    \foreach \x in {0,...,\the\numexpr\worldwidth - 1\relax}{
            \foreach \y in {0,...,\the\numexpr\worldheight - 1\relax}{
                    \node[fill=white, fill opacity=.6,text opacity=1,ellipse, inner sep = 0pt] at (0cm + \tilesize * \x ,0cm + \tilesize * \y ){\tiny\ttfamily\fontseries{b}\selectfont(\x,\y)};
                }
        }
\end{FOPBotWorld}\bigskip\par
Layering example:
\begin{FOPBotWorld}{11}{5}
    \foreach \x/\y in {
            {0/0},
            {0/1},
            {0/2},
            {0/3},
            {0/4},
            {1/4},
            {2/4},
            {1/2},
            {2/2},
            {4/0},
            {4/1},
            {4/2},
            {4/3},
            {4/4},
            {5/4},
            {6/4},
            {6/3},
            {6/2},
            {6/1},
            {6/0},
            {5/0},
            {8/0},
            {8/1},
            {8/2},
            {8/3},
            {8/4},
            {9/4},
            {10/4},
            {10/3},
            {10/2},
            {9/2},
        }{
            \putcoin{\x}{\y}{1}
        }
    \path (5,2) pic {Trianglebot};
    \draw[violet,{Latex}-, ultra thick] (n-5-2) --++(.5,3) node[above]{\sffamily Center};
    \begin{pgfonlayer}{board}
        \draw[left color= yellow, right color = green,opacity=.3,draw=none] ([xshift=.6pt,yshift=.6pt]n-0-0.south west) rectangle ([xshift=-.6pt,yshift=-.6pt]n-\the\numexpr\worldwidth-1\relax-\the\numexpr\worldheight-1\relax.north east);
        % Koordinaten
        \foreach \x in {0,...,\the\numexpr\worldwidth - 1\relax}{
                \foreach \y in {0,...,\the\numexpr\worldheight - 1\relax}{
                        \node[fill=white, fill opacity=.6,text opacity=1,star, inner sep = -2.5pt] at (0cm + \tilesize * \x ,0cm + \tilesize * \y ){\tiny\ttfamily\fontseries{b}\selectfont(\x,\y)};
                    }
            }
    \end{pgfonlayer}
    %Shadows
    \begin{pgfonlayer}{background}
        \draw[draw=none, drop shadow] ([xshift=.6pt,yshift=.6pt]n-0-0.south west) rectangle ([xshift=-.6pt,yshift=-.6pt]n-\the\numexpr\worldwidth-1\relax-\the\numexpr\worldheight-1\relax.north east);
    \end{pgfonlayer}
\end{FOPBotWorld}

\clearpage
Chess Example: (\texttt{TODO: ADD CHESS FIGURES})\bigskip
\begin{FOPBotWorld}{8}{8}
    % König, oder so
    \path (4,4) pic[] {Trianglebot=black};
    \path (3,3) pic[] {Trianglebot=gray};
    % Schachmuster
    \begin{pgfonlayer}{board}
        \draw[fill=fopbot@chess_dark_gray,draw=none] ([xshift=.6pt,yshift=.6pt]n-0-0.south west) rectangle ([xshift=-.6pt,yshift=-.6pt]n-\the\numexpr\worldwidth-1\relax-\the\numexpr\worldheight-1\relax.north east);
        \foreach \x in {0, 2,...,\the\numexpr\worldwidth - 1\relax}{
                \foreach \y [evaluate= \y as \xoffset using {Mod(\y,2) *1cm}] in {0,...,\the\numexpr\worldheight - 1\relax}{
                        \node[fill=white, draw=fopbot@dark_gray, very thick, text width=\tilesize,text height=\tilesize, inner sep = 0pt] at (0cm + \tilesize * \x +\xoffset,0cm + \tilesize * \y ){};
                    }
            }
        \foreach \x in {3,4,5}{
                \foreach \y in {3,4,5}{
                        \node[fill=green, draw=fopbot@dark_gray, very thick, text width=\tilesize,text height=\tilesize, inner sep = 0pt] at (\x,\y){};
                    }
            }
        \node[fill=yellow, draw=fopbot@dark_gray, very thick, text width=\tilesize,text height=\tilesize, inner sep = 0pt] at (4,4){};
        \node[fill=red, draw=fopbot@dark_gray, very thick, text width=\tilesize,text height=\tilesize, inner sep = 0pt] at (3,3){};
    \end{pgfonlayer}
\end{FOPBotWorld}
\end{document}