\documentclass[
    12pt,
    a4paper,
    ngerman,
    leqno
    ]{article}
\usepackage[utf8]{inputenc}
\usepackage[T1]{fontenc}
\usepackage[ngerman]{babel}
\usepackage{tikz}
\usepackage{xifthen}
\usetikzlibrary{patterns, shapes, intersections, arrows, math, decorations,decorations.pathreplacing, decorations.pathmorphing, positioning, calc, automata, chains,matrix, arrows.meta, , shadows.blur, shapes.symbols, backgrounds}
\usepackage{../FOPBot}
\begin{document}
\section*{FOPBot Examples}
Simple Example:
\begin{FOPBotWorld}{3}{3}
    % Walls (Since Noce anchors are for some reason not as exact we use coordinates)
    \draw[very thick] (.5,-.5) -- (.5,1.5);
    \draw[very thick] (1.5,.5) -- (1.5,2.5);
    % Coins
    \putcoin{1}{2}{2}
    \putcoin{1}{0}{2}
    \putcoin{1}{0}{2}
    % Bots
    \path (1,1) pic {Trianglebot};
    \path (0,2) pic[rotate=270] {Trianglebot=red};
    \path (2,2) pic[rotate=180] {Trianglebot=green};
    \path (2,0) pic[rotate=90] {Trianglebot=yellow};
    \path (0,0) pic[] {Trianglebot=violet};
\end{FOPBotWorld}
\vspace{1cm}\\
Example using Loops and a figure:
\begin{figure}[h] 
    \centering
    \begin{FOPBotWorld}{11}{5}
        \foreach \x/\y in {
                {0/0},
                {0/1},
                {0/2},
                {0/3},
                {0/4},
                {1/4},
                {2/4},
                {1/2},
                {2/2},
                {4/0},
                {4/1},
                {4/2},
                {4/3},
                {4/4},
                {5/4},
                {6/4},
                {6/3},
                {6/2},
                {6/1},
                {6/0},
                {5/0},
                {8/0},
                {8/1},
                {8/2},
                {8/3},
                {8/4},
                {9/4},
                {10/4},
                {10/3},
                {10/2},
                {9/2},
            }{
                \putcoin{\x}{\y}{1}
            }
        \path (5,2) pic {Trianglebot};
    \end{FOPBotWorld}
    \caption{Gewünschtes Ergebnis}
\end{figure}
\vspace{1cm}\\
Example using the defined lables: (\texttt{TODO: FIX LAYERS})
\begin{FOPBotWorld}{3}{3}
    % Walls (Since Noce anchors are for some reason not as exact we use coordinates)
    \draw[very thick] (.5,-.5) -- (.5,1.5);
    \draw[very thick] (1.5,.5) -- (1.5,2.5);
    % Coins
    \putcoin{1}{2}{2}
    \putcoin{1}{0}{2}
    \putcoin{1}{0}{2}
    % Bots
    \path (1,1) pic {Trianglebot};
    \path (0,2) pic[rotate=270] {Trianglebot=red};
    \path (2,2) pic[rotate=180] {Trianglebot=green};
    \path (2,0) pic[rotate=90] {Trianglebot=yellow};
    \path (0,0) pic[] {Trianglebot=violet};
    % Using the defined Lables
    \draw[ultra thick, {Latex}-, orange] (n-1-1.center) -- ++ (2cm,0) node[right, font=\sffamily]{center};
\end{FOPBotWorld}
\clearpage
Tetris Example:
\begin{FOPBotWorld}{6}{8}
    %\node[] at (0,0) {\includegraphics[height=.85cm]{orange.png}};
    \path (0,0) pic[rotate=90] {Tetrisbot=yellow};
    \path (0,1) pic[rotate=90] {Tetrisbot=yellow};
    \path (0,2) pic[rotate=90] {Tetrisbot=yellow};
    \path (1,2) pic[rotate=180] {Tetrisbot=green};
    \path (2,2) pic[rotate=180] {Tetrisbot=green};
    \path (2,3) pic[rotate=180] {Tetrisbot=green};
    \path (3,3) pic[rotate=180] {Tetrisbot=green};
    \path (3,4) pic {Tetrisbot=red};
    \path (3,5) pic {Tetrisbot=red};
    \path (2,5) pic {Tetrisbot=red};
    \path (4,4) pic {Tetrisbot=red};
    \path (1,0) pic[rotate=180] {Tetrisbot=blue};
    \path (2,0) pic {Tetrisbot=aqua};
    \path (3,0) pic {Tetrisbot=aqua};
    \path (2,1) pic {Tetrisbot=aqua};
    \path (3,1) pic {Tetrisbot=aqua};
    \path (4,0) pic[rotate=270] {Tetrisbot};
    \path (4,1) pic[rotate=270] {Tetrisbot};
    \path (4,2) pic[rotate=90] {Tetrisbot=purple};
    \path (5,1) pic[rotate=90] {Tetrisbot=purple};
    \path (5,2) pic[rotate=90] {Tetrisbot=purple};
    \path (5,3) pic[rotate=90] {Tetrisbot=purple};
\end{FOPBotWorld}
\end{document}